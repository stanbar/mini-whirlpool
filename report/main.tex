\documentclass[runningheads,a4paper]{llncs}
\usepackage[utf8]{inputenc}
\usepackage{hyperref}
\usepackage{amsmath}
\usepackage{listings}

\title{Whirlpool}
\author{Stanisław Barański}
\date{May 2021}
\institute{Podstawy kryptografi}
\begin{document}

\maketitle

\section{Introduction}

The goal of this project is to find a preimage for the output of the Whirlpool-like hash function. Formally, finding $x$ for defined upfront $h=H(x)$, where $H$ is a Whirpool-like function—modification of the original Whirlpool~\cite{stallings2006whirlpool} function.

\section{Whirlpool}

Whirlpool is a hash function with a cipher-block-based compression function, meaning that its hash code results from encrypting message blocks using a recursive scheme. The output of encryption becomes an encryption key for the next block, the so-called chaining variable.

Whirlpool compression function is an encryption algorithm based on AES; it takes a 512-bit block of plaintext and a 512-bit key as input and produces a 512-bit block of ciphertext as output. The encryption algorithm involves the use of four transformations: Add Key (AK), Substitute Bytes (SB), Shift Columns (SC), and Mix Rows (MR)

This Whirlpool-like function differs from the original Whirlpool by using different padding, smaller hash code (128-bit over 512-bit), different irreducible polynomial (0x12B over 0x011D), different S-box, Diffusion Matrix, and Round Keys.

\section{Usage}

In order to compile and execute the program, Rust toolchain (cargo) needs to be installed. The easiest way to install it is through \href{https://rustup.rs/}{rustup.rs}. In order to run the program in debug mode execute
\begin{verbatim}
    cargo run
\end{verbatim}
it will expect the input string on stdin. Or just
\begin{verbatim}
    echo -n Hello World | cargo run
\end{verbatim}
Alternatively the input can be passed as an argument
\begin{verbatim}
    cargo run -- "Hello World"
\end{verbatim}

In order to execute the project goal (finiding the preimages) execute
\begin{verbatim}
    cargo run --bin reverse-hash --release
\end{verbatim}

\section{Implementation}
I decided to implement the program in \href{https://en.wikipedia.org/wiki/Rust_(programming_language)}{Rust}—relatively new (released on July 7, 2010) programming language. Rust offers C-level performance, a borrow checker, an excellent type system, and a modern toolchain, making it \href{https://insights.stackoverflow.com/survey/2020#most-loved-dreaded-and-wanted}{the most loved programming language of 2020 (according to StackOverflow 2020 Developer Survey}. It is also a popular choice for new projects where cryptography is involved.

Besides std, I used one external library(\href{https://github.com/rayon-rs/rayon}{Rayon}) to achieve easy multi-threaded execution.

\section{Multiplication in GF field}
Efficient modulo multiplication in GF was the most challenging part of the implementation.
I started with the most straightforward implementation of multiplying two polynomials.  
\[
c = (\sum{a_i x^i}) \otimes (\sum{b_j x^j}) = \bigoplus_{i=0}\bigoplus_{j=0} (a_i \otimes b_i) x^{i+j}
\]
then calculating $c \pmod{x^8 + x^5 + x^3 + x + 1}$ using Long Polynomial Division and taking the remainder. It worked but required a lot of computations.

First, I noticed multiplying two polynomials of degree up to 7 results with a polynomial of degree up to 14.

For example, multiplying two polynomials (FF) $\otimes$ (FF) equals
\begin{align*}
&(x^7  + x^6  + x^5  + x^4  + x^3  + x^2 + x^1 + x^0) \otimes (x^7 + x^6 + x^5 + x^4 + x^3 + x^2 + x^1 + x^0) \\
& = (x^{14} + x^{13} + x^{12} + x^{11} + x^{10} + x^9 + x^8 + x^7) \\
& \oplus      (x^{13} + x^{12} + x^{11} + x^{10} + x^9 + x^8 + x^7 + x^6) \\
& \oplus             (x^{12} + x^{11} + x^{10} + x^9 + x^8 + x^7 + x^6 + x^5) \\
& \oplus                    (x^{11} + x^{10} + x^9 + x^8 + x^7 + x^6 + x^5 + x^4) \\
& \oplus                           (x^{10} + x^9 + x^8 + x^7 + x^6 + x^5 + x^4 + x^3) \\
& \oplus                                  (x^9 + x^8 + x^7 + x^6 + x^5 + x^4 + x^3 + x^2)\\
& \oplus                                        (x^8 + x^7 + x^6 + x^5 + x^4 + x^3 + x^2 + x^1) \\
& \oplus                                              (x^7 + x^6 + x^5 + x^4 + x^3 + x^2 + x^1 + x^0) \\
& = x^{14} + x^{12} + x^{10} + x^8 + x^6 + x^4 + x^2 + 1
\end{align*}

Therefore I hardcoded the multiplications reducing the execution time by -50\%. 

\begin{lstlisting}
// arr and brr are first and second polynomials
encoded as arrays of coefficients.
// out is the resulting polynomial.

        out[14] = arr[7] & brr[7];
        out[13] = (arr[7] & brr[6]) ^ (arr[6] & brr[7]);
        out[12] = (arr[7] & brr[5]) ^ (arr[5] & brr[7]) ^ (arr[6] & brr[6]);
        out[11] = (arr[7] & brr[4]) ^ (arr[4] & brr[7]) ^ (arr[6] & brr[5]) ^ (arr[5] & brr[6]);
        out[10] = (arr[7] & brr[3])
            ^ (arr[3] & brr[7])
            ^ (arr[6] & brr[4])
            ^ (arr[4] & brr[6])
            ^ (arr[5] & brr[5]);
        out[9] = (arr[7] & brr[2])
            ^ (arr[2] & brr[7])
            ^ (arr[6] & brr[3])
            ^ (arr[3] & brr[6])
            ^ (arr[5] & brr[4])
            ^ (arr[4] & brr[5]);
        out[8] = (arr[7] & brr[1])
            ^ (arr[1] & brr[7])
            ^ (arr[6] & brr[2])
            ^ (arr[2] & brr[6])
            ^ (arr[5] & brr[3])
            ^ (arr[3] & brr[5])
            ^ (arr[4] & brr[4]);
        out[7] = (arr[0] & brr[7])
            ^ (arr[7] & brr[0])
            ^ (arr[6] & brr[1])
            ^ (arr[1] & brr[6])
            ^ (arr[5] & brr[2])
            ^ (arr[2] & brr[5])
            ^ (arr[3] & brr[4])
            ^ (arr[4] & brr[3]);
        out[6] = (arr[0] & brr[6])
            ^ (arr[6] & brr[0])
            ^ (arr[5] & brr[1])
            ^ (arr[1] & brr[5])
            ^ (arr[4] & brr[2])
            ^ (arr[2] & brr[4])
            ^ (arr[3] & brr[3]);
        out[5] = (arr[0] & brr[5])
            ^ (arr[5] & brr[0])
            ^ (arr[4] & brr[1])
            ^ (arr[1] & brr[4])
            ^ (arr[3] & brr[2])
            ^ (arr[2] & brr[3]);
        out[4] = (arr[0] & brr[4])
            ^ (arr[4] & brr[0])
            ^ (arr[3] & brr[1])
            ^ (arr[1] & brr[3])
            ^ (arr[2] & brr[2]);
        out[3] = (arr[0] & brr[3]) ^ (arr[3] & brr[0]) ^ (arr[2] & brr[1]) ^ (arr[1] & brr[2]);
        out[2] = (arr[0] & brr[2]) ^ (arr[2] & brr[0]) ^ (arr[1] & brr[1]);
        out[1] = (arr[0] & brr[1]) ^ (arr[1] & brr[0]);
        out[0] = arr[0] & brr[0];
\end{lstlisting}

I noticed that calculating modulo in finite field can be achieved by simple substitutions that hold congruence property over irreducible polynomial $p$ (in my case $x^8 + x^5 + x^3 + x + 1$).

\begin{align*} 
x^8 + x^5 + x^3 + x + 1 &\equiv 0 \pmod{p}\\
x^8 &\equiv x^5 + x^3 + x + 1 \pmod{p}
\end{align*} 

Whenever polynomial $c$ contains monomial $x^8$, it can be replaced with polynomial $x^5 + x^3 + x + 1$ while still holding congruence property.

The same can be done with all monomials of degree $> 8=degree(p)$
\begin{align*} 
x^9 = x^8 * x = (x^5 + x^3 + x + 1) * x &\equiv x^6 + x^4 + x^2 + x \pmod{p} \\
x^{10} = x^9 * x = (x^6 + x^4 + x^2 + x) * x &\equiv x^7 * x^4 * x^3 + x^2 \pmod{p} \\
x^{11} = x^{10} * x = (x^7 * x^4 * x^3 + x^2) * x = x^8 + x^5 + x^4 + x^3 &\equiv x^4 + 1 \pmod{p} \\
x^{12} = x^{11} * x = (x^4 + 1) * x &\equiv x^5 + x \pmod{p} \\
x^{13} = x^{12} * x = (x^5 + x) * x &\equiv x^6 + x^2 \pmod{p} \\
x^{14} = (x^6 + x^2) * x &\equiv x^7 + x^3 \pmod{p}
\end{align*} 

By using these substitutions, I reduced the execution time dramatically, about -80\%.
\begin{lstlisting}
// x^8 = x^5 + x^3 + x + 1
if out[8] {
    out[5] ^= true;
    out[3] ^= true;
    out[1] ^= true;
    out[0] ^= true;
}

// x^9 = x^6 + x^4 + x^2 + x
if out[9] {
    out[6] ^= true;
    out[4] ^= true;
    out[2] ^= true;
    out[1] ^= true;
}
// x^10 = x^7 + x^4 + x^3 + x^2
if out[10] {
    out[7] ^= true;
    out[4] ^= true;
    out[3] ^= true;
    out[2] ^= true;
}
// x^11 = x^4 + 1
if out[11] {
    out[4] ^= true;
    out[0] ^= true;
}
// x^12 = x^5 + x
if out[12] {
    out[5] ^= true;
    out[0] ^= true;
}
// x^13 = x^6 + x^2
if out[13] {
    out[6] ^= true;
    out[2] ^= true;
}
// x^14 = x^7 + x^3
if out[14] {
    out[7] ^= true;
    out[3] ^= true;
}
\end{lstlisting}


\section{Results}
The attacks were performed on my server machine equipped with a processor Intel i7-4790 (8) @ 4.000GHz, 16GB (2*8GB) of DDR3 RAM, running Arch Linux operating system with kernel version 5.11.8. For compilation I used cargo 1.51.0 (43b129a20 2021-03-16).


\begin{table}
\centering
\begin{tabular}{||c | c | c||} 

 \hline
Input length & Preimage found & Execution time   \\ [0.5ex] 
 \hline\hline
2            & 0J             & 25.635827ms      \\
3            & Ss1            & 1.89898985s      \\
4            & @@-8           & 172.601573451s   \\
5            & qDP0Z          & 10431.405587416s
   \\
6            & Not Found          & Not Found \\
\hline
\end{tabular}
\end{table}


Unfortunately, I did not manage to find a preimage for six-digit input over five days. Based on the five-digit attack, the average time for calculating one hash took $10431.405587416s/79^5  =0.00000339005s$. Therefore six-digit attack could take $79^6 * 0.00000339005s = 824078.628589s = 824078.628589s = 9.53794709015days$.


\bibliographystyle{splncs04}
\bibliography{references}
\end{document}
